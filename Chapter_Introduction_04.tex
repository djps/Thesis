\chapter{Introduction} \label{chap:intro}
%
\par % Introductory sentence
The problem of the configurations and buckling of elastic conducting wire in a magnetic field is a classical one in magnetoelasticity and is of both theoretical and practical interest. In this thesis the problem is investigated using modern techniques from dynamical systems theory. %
%
\par % Background of magnetoelasticity
It is well-known that a straight current-carrying wire held between pole faces of a magnet buckles into a coiled configuration at a critical current~\cite[\S{10.4.3}]{Woodson68}. A rigorous bifurcation analysis of this buckling problem (for a uniform magnetic field directed parallel to the undeformed wire) was developed in a series of papers by Wolfe. Wolfe first considered a nonlinearly-elastic string model for the wire, i.e., a perfectly flexible elastic line, and by studying the linearised eigenvalue problem about the trivial straight solution found that an infinite number of solution branches bifurcate from the trivial solution~\cite{Wolfe83,Wolfe90b}, much like in the Euler elastica under compressive load. Indeed, the analysis was performed in a classical manner, that is a configuration space was defined using Euler angles and the centreline of the rod and through the construction of potential energies a set of Euler-Lagrange equations produced. It was then shown that the equations can be solved exactly and the non-trivial solutions are exact helices. In subsequent work Wolfe modelled the wire as a rod~\cite{Wolfe88,Seidman88}. In addition to extension a rod can undergo flexure, torsion and shear, and for the case of welded boundary conditions it was found that in certain cases bifurcation occurs, again with an infinity of non-trivial equilibrium states. The bifurcation results were obtained through an imperfect bifurcation since the solution branches were found to be two-manifolds. Healey~\cite{Healey90} then performed the bifurcation analysis for a wire by observing that a `subtle symmetry', due to reversibility through the rod, meant that the solution branches were two-dimensional manifolds.  In fact, the observation allowed the problem to be reduced to a form that was amenable to global bifurcation analysis~\cite{Healey88}. Wolfe then performed the same analysis for an elastic rod in a uniform magnetic field~\cite{Wolfe96}. 
% 
\par % Motivation, modern context
Many technical devices such as motors, generators and transformer use elastic structures in magnetic fields~\cite{Moon84} but recently the problem of a conducting rod in a magnetic field has attracted interest as a model for electrodynamic space tethers~\cite{Valverde03,Heijden05}. Electrodynamic tethers are long slender conducting cables that exploit the earth's magnetic field to generate Lorentz forces through Faraday's law. The generated drag force could be used for maneuvering satellites when deorbiting, eliminating the need for additional chemical fuel, thus reducing the weight of satellite and hence operational costs. The reduction in cost has been estimated at a billion dollars over ten years for the international space station alone~\cite{Johnson98}. Tethers are spun about their axis for gyroscopic stability and therefore must resist bending and twisting. Such tethers need to be described by an elastic rod rather than the traditional wire. Analysis of electrodynamic tethers has been performed using techniques from multibody system dynamics.  Geometric nonlinearities were found to have a stabilizing effect on the tether configurations. However, a drawback with the analysis was that elastic displacements in each substructure were assumed to be small, diminishing the stabilizing effect. %
%
\par % Hamiltonian formulation
It is believed that modern problems in engineering need modern solutions and in this thesis a \emph{geometrically exact} formulation is adopted. The static equilibrium equations of a rod under end force and moment, known as the Kirchhoff equations~\cite{Antman95} are extended by incorporating a fixed external vector in the direction of the magnetic field into the body frame that interacts with the rod via a Lorentz force. The geometrically exact formulation, in contrast to previous models, is naturally a noncanonical Hamiltonian formulation~\cite{Simo88} and retains the symmetry properties of the physical system. The Hamiltonian formulation allows deep insight into a system and allows a number of powerful methods to be applied; for example in the study of nonlinear stability~\cite{Simo90,Chouaieb04}; bifurcation theory~\cite{Cushman90,Chossat02}; complete integrability~\cite{Arnold89,Kehrbaum97a}; spatially chaotic solutions~\cite{Devaney78,Holmes83} and in numerical analysis~\cite{Channell90}. A principal advantage is the large body of work relating to finite-dimensional noncanonical Hamiltonian systems~\cite{Arnold89,Marsden99}. % 
% 
\par % In general: applications of rod theory
When the rod is isotropic, that is when the principal bending stiffness are equal, the Kirchhoff equations are completely integrable and all solutions can be expressed in closed form. As such, the Kirchhoff equations have been used to model a variety of physical systems. Examples include: the deformation of biological materials such as DNA~\cite{Naschie90,Schlick95}, climbing plants~\cite{Goriely98b,McMillen02}, the visualisation of hair~\cite{Bertails05}, the spin dynamics of the superfluid~ ${}^{3}$He~\cite{Novikov82b} and an Heisenberg $XY$ particle~\cite{Dandoloff05}, the configurations of undersea cables~\cite{Coyne90}, the motion of a body submerged in an ideal compressible fluid~\cite{Smith01,Holmes98,Leonard97a,Leonard97b}. The Kirchhoff equations are related to the (integrable) modified Korteweg-de Vries equation by the Hasimoto transform~\cite{Hasimoto71,Langer96} and through the Kirchhoff kinetic analogy to the vast canon of literature devoted to the motion of rigid bodies~\cite{Marsden99,Wang91}. %
% 
\par % Kirchhoff kinetic analogy
Despite the Kirchhoff equations being static, mathematically they have the same structure as many problems in dynamics: arclength along the rod plays a role similar to that of time in a dynamical system like the spinning top or a pendulum. The Kirchhoff kinetic analogy relates the \emph{shape} of a deformed rod with the \emph{motion} of a heavy spinning top~\cite{Thompson88a,Kehrbaum97b,Coyne90}. In the same way the motion of the centre of gravity of a top prescribes the motion of the entire top so the position of the centreline of the rod prescribes the position of the rigidly transformed cross-section. For example, an initially straight rod whose principal moments of inertia are equal can be deformed via end forces and moments into a helix, corresponding to the periodic orbit of the spinning top. The analogy is not perfect however, as concepts such as shear, extensibility and nonlinear constitutive relationships have no physical interpretation in the context of rigid body dynamics\footnote{In many ways the rod model is more flexible than the rigid body model!}. % 
%
\par % localisation as homoclinic solutions.
Homoclinic solutions represent localising buckling modes which are the physically preferred buckling configurations for long rods~\cite{Thompson96,Horak05} and thus are the natural configurations to study. In general homoclinic solutions to a hyperbolic fixed point are a codimension-one phenomena, however in Hamiltonian or reversible systems homoclinic solutions are a codimension-zero phenomena and hence are generic under perturbations~\cite{Devaney76b}. Homoclinic solutions disregard the intricacies and influences of boundary conditions and can be thought of as a purer form of buckling. Homoclinic buckling of rods due to magnetic effects has not been investigated up to this point. %
%
\par % rods, specifically material properties
Previous analysis of localised solutions of Cosserat rods under end torque and moment has shown that neither shear and extensibility~\cite{Stump00,Champneys97a} nor nonlinear constitutive relationships~\cite{Champneys96b,Antman75} has any significant qualitative or quantitative effect on the rod. Indeed, in both cases the isotropic system was integrable. Other material properties such as anisotropy~\cite{Mielke88} and initially curvature~\cite{Leung04} lead to the loss of complete integrability and the emergence of spatial chaos resulting in multimodal configurations. The resulting localised configurations and their bifurcation structure were investigated in~\cite{Heijden98a} and~\cite{Champneys97a} respectively. Nonlinear normal form analysis was performed on the buckling of anisotropic~\cite{Heijden98b} and initially curved rods~\cite{McMillen02}. It was shown that a codimension-two point delineates between weakly anisotropic rods and strongly anisotropic rods. Weakly anisotropic rods buckle according to a subcritical Hamiltonian-Hopf bifurcation, strongly anisotropic rods buckle according to the Hamiltonian-pitchfork bifurcation~\cite{Heijden98b}.  
%
\par % family of noncanonical models
In this thesis material properties are not the main focus of the investigation, instead the governing equations are extended to include the effect of the magnetic field. It is shown that the static equilibrium equations for a rod in a magnetic field sit in a family of non-canonical Hamiltonian systems. The first member of the family is the force-free rod (the Euler-Poinsot top), the Kirchhoff equation is the second member (the Lagrange top) and the third member the rod in a magnetic field (the abstract Twisted top~\cite{Thiffeault01}). The fourth member of the family is a rod in a linearly varying magnetic field that depends on the configuration of the rod. It will be shown that the rod in a uniform magnetic field is significant as it is the first member whose Lie-Poisson bracket is extended in a nontrivial manner. %
%
\par % Lax pairs
Crucially, every member of the family is completely integrable in the sense of Liouville~\cite{Arnold89} if the constitutive relationships are linearly elastic, initially straight, isotropic, unshearable and inextensible. The subfamily can be generalised using a Lax pair formulation~\cite{Sinden08}. %
% 
\par % thesis specific reduction
The Euler angles are used to reduce the governing equation for a rod in a uniform magnetic field from a nine-dimensional non-canonical Hamiltonian system to a six-dimensional canonical system. The reduction holds provided that the magnetic field is not aligned with the rod at any point as the system loses rank~\cite{Olver93}. Due to the nontrivial Poisson bracket extension, the Casimirs and integrals that render the system completely integrable do not have an intuitive physical interpretation. Thus, reduction to a single degree of freedom system is seemingly impossible. Hence, when the rod is isotropic the system is reduced to a four-dimensional system with an additional integral. However, from the four-dimensional system phase diagrams are computed as planar projections of Poincar\'e sections of a three-dimensional energy surface. %
% 
\par % rod specific: Mel'nikov
From the reduction a modified version of Mel'nikov's method that takes into account the symmetry of the reduced system~\cite{Holmes83} is applied to show that the presence of a uniform magnetic field is a perturbation which destroys integrability for extensible rods. The Mel'nikov analysis shows that the stable and unstable manifolds of the perturbed homoclinic orbit intersect transversely and there exists Smale horseshoes on the Poincar\'e sections of the homoclinic energy level~\cite{Smale67,Devaney78}. The loss of integrability is the classical result but it is the presence of horseshoes which is the more significant result: given the presence of horsehoes on the stable and unstable manifolds there exists a multiplicity of multimodal homoclinic orbits~\cite{Belyakov90}. %
% 
\par % differences in old model to new
The result is of interest as alone neither the presence of a magnetic field nor extensibility effect the integrability of a rod. However, when extensibility and the magnetic field are both considered to be perturbations of equal order the interaction of the combined perturbations destroys integrability. It is believed that this is the first physical system in which the coupling between two integrable perturbations leads to the loss of integrability and spatially chaotic solutions. %
% 
\par % thesis specifics - periodicity
Due to the coupling of body and spatial frames by the magnetic field the computation and continuation of homoclinic orbit are nontrivial since the homoclinic orbit are periodic-to-periodic~\cite{Champneys97d,Bai96a}. The coupling between body and spatial frames has been seen in the context of rods constrained to lie in a plane~\cite{Heijden99} or on a cylinder~\cite{Heijden01a,Heijden02b} and standard numerical procedures are adapted. The underlying periodicity does not affect the codimension of the problem. Hence localised solutions are computed and their post-buckling paths followed using the continuation software~\textsc{auto97}~\cite{Doedel98}. However, it is impossible to derive analytical expressions for the spectrum of Floquet multipliers. 
%
\par % thesis specifics - bif structure
A rich bifurcation structure for primary homoclinics and bi- and tri-modal homoclinics is uncovered. A codimension-two point is identified from the spectrum of the Floquet multipliers which determines whether configuration with a single localisation bifurcate twice, once or not at all for critical values of the magnetic field. The bifurcation is found to be a twice generalised Hopf bifurcation; generalised by the Hamiltonian structure and the periodicity of the trivial solution. Consequently the bifurcation can be called a Hamiltonian Hopf bifurcation about a periodic orbit or equivalently a Hamiltonian Neimark-Sacker bifurcation. The codimension-two point is a double Hamiltonian-Hopf bifurcation point and acts as an organising centre for the bifurcation set for primary and multimodal homoclinics. 
%
\par % Outline - chapters
The thesis is structured as follows, chapter~\ref{chap:poisson} gives the theoretical background and outlines the analytical tools used in thesis. Chapter~\ref{chap:model} then derives the governing equations of geometrically exact rod as a non-canonical Hamiltonian system under a variety of loading conditions. A family of equilibrium equations is then formulated. For a set of constitutive relations a Lax pair presented. Chapter~\ref{chap:reduction} gives a description of the Kirchoff rod. In chapter~\ref{chap:analytical} the rod in a magnetic field is reduced on the symplectic leaves to form a canonical system. Mel'nikov's method is then applied to an extensible rod in magnetic field. From the existence of multimodal solutions chapter~\ref{chap:bifurcation} investigates the multiplicities of multimodal homoclinics and their bifurcation structure. Finally, chapter~\ref{chap:conc} summarises the main results, discusses conclusions, limitations and directions for future research. The thesis concludes with two appendices; the first introduces the Euler angles and the Euler parameters, the second outlines the standard numerical techniques to compute and continue homoclinic orbits in reversible dynamical systems to a hyperbolic fixed point. %