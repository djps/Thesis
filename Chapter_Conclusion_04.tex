%---------------------------------------------
%
% The Rule of three: this is the last part.
%
%---------------------------------------------
%
\chapter{Conclusion} \label{chap:conc}
%
\par % Description of this chapter
The objective of this thesis was to investigate the behaviour of a conducting elastic rod in a magnetic field and a number of original contributions have been made. Firstly, the identification of the static equilibrium equations as a noncanonical Hamiltonian system which for a class of constitutive relations is completely integrable. Secondly, through detailed perturbation analysis it is shown that it is the direct coupling of the magnetic effects and extensibility leads to spatial chaos. Thirdly, the existence of a codimension-two Hamiltonian Hopf-Hopf bifurcation and its role as an organising centre for the nearby dynamics.  %  
%
\par % Discussion - Hamiltonian formulation
Perhaps the most important step in the analysis was recognising that the static equilibrium equations were a noncanonical system as the Hamiltonian structure is emphasised and exploited throughout this thesis. For example, the Hamiltonian structure is necessary in order to prove complete integrability in the unperturbed system, in the reduction to a canonical system, in the Mel'nikov analysis and, by exploiting the codimension of homoclinic solutions in Hamiltonian systems, to produce load-deflection diagrams.
%
\par % Summary - integrability
The equilibrium equations are, like the Kirchhoff equations, Lie-Poisson equations. It is shown that the new Lie-Poisson bracket is produced via Liebniz and semidirect extensions to the Kirchhoff bracket. The bracket extensions are generalised and the equilibrium equations found to sit, as the third member, in a family of rod equations in generalised magnetic fields. Thus, the rod in a uniform magnetic field gives a physical realisation to the abstract `Twisted top'~\cite{Thiffeault01}. Interestingly, the Hamiltonian remains unchanged and effect of the magnetic field is contained within the structure matrix. A new and rather pleasing observation is that the contributions of each generalised force on the rod is a provided by a nontrivial bracket extension. %
% 
\par % Summary - bracket
However, it is the form of the Poisson bracket which presents a major obstacle in the analysis. A complete interpretation of all of the symmetry laws which determine the integrals would allow, through a suitable parameterisation, a reduction to a single degree of freedom system. Such a reduction would then allow the homoclinic orbits to be expressed in closed form, the application of Mel'nikov analysis when extensibility is the perturbation parameter, action-angle variables to be constructed and closed form expressions for the buckling loads. % For Lie-Poisson equations with a symmetry law, using infinitesimal generators on a suitable inner product, conserved quantities can be found~\cite{Simo90}. However, there is no algorithmic procedure which from a conserved quantity associates a symmetry. %
% 
\par % Discussion - structure of the force
The equilibrium equations are shown to be integrable for a class of constitutive relations. Thus, when a rod is ordinarily straight, linearly elastic, isotropic, inextensible and unshearable and a generalised force of the form $\boldsymbol{F}\times\boldsymbol{d}_{3}$, i.e., some field $\boldsymbol{F}$ acts normal to the rod, then the resulting system is integrable. The integrable subfamily of equations is then parameterised by a Lax pair. 
%
\par % Discussion - Hidden conditions of the Lax pair
The Lax pair formulation reveals hidden conditions on the constitutive relations in order for all the members of the family of rod equations to be integrable. For example, the force-free is (super)integrable regardless of isotropy or nonlinear constitutive relations, in contrast to the Kirchhoff rod which requires isotropy in order to be integrable. The rod in a magnetic field now requires the addition conditions of inextensibility, unshearability and linear elasticity along with isotropy in order to be integrable, in contrast to the two previous members. %
% 
\par % Discussion - Casimirs and first integrals in Lax pair
It is interesting to note that in the Lax pair formulation some Casimirs are `promoted' to first integrals (and thus become conditional on the constitutive relations) as a new field is added in going to the next `generation' of the family. For instance, at the second level of the family $\mathsf{n}$ is added as a uniform field and hence $\frac{1}{2}\mathsf{n}\cdot\mathsf{n}$ is a Casimir. In the next perturbation, by the field $\mathsf{B}$, the Casimir is perturbed to $\frac{1}{2}\mathsf{n}\cdot\mathsf{n}+\mathsf{m}\cdot\mathsf{B}$. After one more perturbation, by the field $\mathsf{D}$, this Casimir is turned into the first integral $\frac{1}{2}\mathsf{n}\cdot\mathsf{n}+\mathsf{m}\cdot\mathsf{B}+B\mathsf{D}\cdot\mathsf{d}_3$. By contrast, the Casimir $\mathsf{n}\cdot\mathsf{m}$ at the second level is perturbed directly into the integral $\mathsf{n}\cdot\mathsf{m}+ B \mathsf{B}\cdot\mathsf{d}_3$ at the next level and remains the same one level up. %
% 
\par % future - superintegrable configs
With the exception of the superintegrable force-free rod, each member of the integrable family has an action-angle formaultaion which exists on an odd-dimensional torus. An isotropic, inextensible rod in a uniform magnetic field exists on a five-torus and through analysis of previous members of the integrable family superintegrable configurations can be classified. For example, configurations on one-tori are either straight twisted rods or untwisted rings, on two-tori configurations are helices and on three-tori configurations are (generically) quasi-periodic helices. However the form of minimally superintegrable configurations on a four-torus remains unknown, indeed localised solutions may exist. Similarly, for a rod in a nonuniform magnetic field the configurations which define six-tori would have additional symmetries. The characterisation of all configurations which define even dimensional tori, with the exception of the helix, remains unknown. %
% 
\par % future - Kovalevskaya
It remains an open question as to whether two Kovalevskaya-type integrals exist for the rod in a magnetic field. In this case an integral that as $\lambda \rightarrow 0$ recovers the Kovalevskaya integral would exist. There is numerical evidence which suggests that at the original condition on the bending stiffnesses no such integral exists~\cite{Thiffeault01} but the condition itself may be perturbed. The form of a prospective second integral remains unknown. A Lax pair formulation does exist for a generalised Kovalevskaya top~\cite{Bobenko89} but, unfortunately the extended model generalises a class of external moments rather than forces. In this formulation the second integral is quartic function of the three field variables~\cite[$\left(3.2\right)$]{Kharlamov05}.  One possible approach to finding a new integrable case would be to replicate Kovalevskaya's original method using a symbolic manipulation package to perform Painlev\'e-Nevanlinna analysis and solve the resulting Diophantine equations. This approach would give the condition on the nondimensional parameters such that the system was integrable but would not reveal the form of the two unknown integrals. %
% 
\par % Reduction
The noncanonical equilibrium equations of an isotropic rod in a uniform magnetic field are reduced on the symplectic leaves defined by the Casimirs to an `autonomous' four-dimensional canonical Hamiltonian system with an integral. By specifying an energy level planar projections of the four-dimensional system yield closed curves. It is shown that if the rod is aligned anywhere with the magnetic field it is aligned everywhere with the field. This is the natural result and, as expected, in this unique case the rod is simply a straight twisted rod.  For small values of the rescaled magnetic field the integrable autonomous four-dimensional system is reduced to an integrable nonautonomous two-dimensional system.
%
\par % Summary - Mel'nikov
Mel'nikov's method was then used to show that for an extensible rod the presence of the magnetic field destroys integrability, leads to Smale horsehoes and the transverse intersections of the stable and unstable manifolds of the homoclinic orbit. Through detailed scaling arguments it is shown that it is the \emph{interaction} between extensibility (a material nonlinearity) and magnetic effects (a geometric nonlinearity) which destroys integrability as neither perturbation alters either the integrability or the transversality of the system.   Specifically, to first order (the sum of the two perturbations) the Mel'nikov function is zero but to second order (the product of the two perturbations) and higher the Mel'nikov function has simple zeroes.  This, it is believed is the first example of such a phenomena in rod mechanics.  Consequently a multiplicity of localised configurations exist. The analysis is complemented by Poincar\'e sections on the homoclinic energy level associated with the loss of integrability. %  
%
\par % Discussion : destruction of a new integral
Previous Mel'nikov analysis had shown anisotropy or initial curvature would destroy the twist integral~\eqref{eq:magnetic_lagrange}. Now, for an extensible rod, it is shown that the effect of the magnetic field preserves the twist integral but destroys the new integral~\eqref{eq:int2}. %
%
\par % Discussion : Mel'nikov and analysis
The Mel'nikov analysis only applies for a small $\delta$-perturbation, while multimodal solutions can be found for all non-zero values of $\lambda$. This is because the Mel'nikov analysis provides a first (or second) order approximation on the splitting of the stable and unstable manifolds for the perturbation based on inverting the Hamiltonian and solving for an expansion of the action integral by the Implicit Function Theorem.  The existence of simple zeroes of the Mel'nikov integral then implies the transversality of the intersection of the stable and unstable manifolds. However, Devaney's theorem, which states that a Hamiltonian system with a transverse point will have a multiplicity of multimodal homoclinics, only requires that the intersection of the stable and unstable manifolds be transverse, which is a global phenomenon for hyperbolic homoclinic orbits in Hamiltonian systems since the intersections will occur along on energy level~\cite{Champneys96b}. %
%  
\par % Discussion : other parameters
As mentioned, closed form solutions of homoclinics for a isotropic inextensible rod in a magnetic field are not known. However, it is believed that a Mel'nikov analysis where extensibility is the perturbation for a rod in a uniform magnetic field would destroy integrability. This is supported by numerical evidence, presented in \S\ref{chap:bifurcation}. Furthermore, it is believed that the effect of a magnetic field on a rod with other material properties, such as nonlinear constitutive relations, would also lead to spatial chaos as linear elasticity, like inextensibility, is a necessary condition on the Lax pair formulation. %
% 
\par % Discussion : isotropy, as integrability breaking parameter
If isotropy is broken then numerical evidence, in the form of multimodal configurations, suggests that Mel'nikov analysis will show that the loss of integrability is accompanied by spatial chaos.  For the rod in a magnetic field two integrals~\eqref{eq:magnetic_integrals} are conditional on isotropy. Thus, Arnol'd diffusion will appear on the Poincar\'e hypersections of the homoclinic energy level~\cite{Holmes82b}. However, any analysis would require the homoclinic configuration for an isotropic rod in a magnetic field, which as mentioned, has yet to be constructed.  Attempting to follow the analysis of~\S\ref{subsec:case1} and rescale $\lambda$ and $\rho$ by a new perturbation parameter $\delta$ so that as the magnetic field increases so does the degree of anisotropy, would be, to first order, no different the Mel'nikov analysis performed in~\S\ref{subsec:anisotropy} which shows simple zeroes. Again, numerical evidence supports the claim that an anisotropic rod in a magnetic field is not integrable.
%
\par % Summary - numerics
Localised solutions were then computed using a three parameter shooting method exploiting the reversibilities of the system. Due to the coupling of the spatial and director frame by the magnetic field standard numerical procedures need to be adapted to deal with the periodicity of the trivial solution. Localised solutions were continued with pseudo arclength continuation software using projection boundary conditions which utilises the exponential trichotomies of the system. The bifurcation structure of the localised solutions with respect to the two nondimensional loading parameters was then investigated. Good agreement was found between the buckling values of primary homoclinics computed from continuation and those predicted through analysis of the Floquet multipliers. The bifurcation was classified as a twice generalised Hopf bifurcation; generalised by the Hamiltonian structure and the by the presence of an underlying trivial periodic orbit. The bifurcation may be described as a Hamiltonian-Hopf bifurcation about a periodic orbit or as a Hamiltonian Niemark-Sacker bifurcation. A codimension-two point was then identified which at which a double Hamiltonian-Hopf bifurcations occurred. The codimension-two point determined whether primary homoclinics bifurcate twice, once or not at all.  The bifurcation structure of multimodal configurations was then investigated. The codimension-two point was then found to be an organising centre for the bifurcation set of the primary and multimodal homoclinics. It is believed that this is the first example of a Hamiltonian-Hopf-Hopf bifurcation seen.
% 
\par % Discussion - Periodicity
A complete understanding of the bifurcation is inhibited by the periodicity of the underlying solution. Consequently nonlinear normal form analysis is not performed. Whether the Hamiltonian-Hopf bifurcation at $\lambda_{+}$ or $\lambda_{-}$ is either subcritical or supercritical is an open question.  Furthermore, insight into the bifurcation at the codimension two point remains unknown. For example, if the bifurcation at $\lambda_{c}$ is transcritical or otherwise, or whether the Floquet multipliers split, pass or cross or whether they behave in an altogether different manner are unknown. 
%
\par % Discussion - normal form of Hamiltonian-Hopf-Hopf
If the trivial solution was a straight and untwisted rod then normal form analysis of the bifurcation could be performed through a Lyapunov-Schmidt reduction of the nine-dimensional governing equations by the three Casimirs following~\cite{Meer90}. An analytical condition on the codimension-two point could then be formulated. However, by reducing the governing equations using the Casimirs the symmetry properties would be preserved and hence the normal form would not be generic. Nor would the nondimensional bifurcation parameters $\lambda$ and $m$ be the natural unfolding parameters which independently determine the bifurcations and the distance between the bifurcations. %
% 
\par % future - accumulation
It would be of interest to generalise the coalescence rules for higher order multimodals in the Kirchhoff rod. Furthermore it would be of interest to investigate whether there exists a set accumulation rules for multimodal homoclinics when the underlying orbit is a periodic orbit and if they exist, how they correspond to those when the underlying orbit is a fixed point. 
% 
\par % Discussion : isotropy and normal form
It is natural to ask whether there exists a codimension-two point determined by critical values $\left(\rho_{c},\lambda_{c}\right)$ which delineates between strongly and weakly anisotropic buckling due to the magnetic field. Following from~\cite{Heijden98b} one may expect that the buckling mechanism for a weakly anisotropic rod is a Hamiltonian-Hopf bifurcation about a periodic solution while in the strongly anisotropic case that the rods buckle in a Hamiltonian-Pitchfork bifurcation about a periodic solution.  %
%
\par % future - other forces
Finally, it is straightforward to show, from the classical construction using the Legendre transform of a Lagrangian, that the equilibrium equations of a heavy rod in a gravitational field are Hamiltonian~\cite{Heijden06a}. Whether the equilibrium equations naturally have a Lie-Poisson formulation is an open question. Indeed, whether a Lie-Poisson formulation can include a variety of body forces or if the model presented here is unique, is another open question. %